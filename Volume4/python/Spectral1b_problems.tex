
Use the differentiation matrix to numerically approximate the derivative of $u(x) = e^{x}\cos(6x)$ on a grid of $N$ Chebychev points where $N=6, 8,$ and $10.$
(Use the linear system $D U \approx U'$.)
Then use barycentric interpolation (\li{scipy.interpolate.barycentric_interpolate}) to approximate $u'$ on a grid of 100 evenly spaced points.

Graphically compare your approximation to the exact derivative.
Note that this convergence would not be occurring if the collocation points were equally spaced.

Use the pseudospectral method to solve the boundary value problem
\begin{align*}
&{ } u'' = e^{2x}, \quad x \in (-1,1), \\
&{ } u(-1) = 0, \quad u(1) = 0.
\end{align*}

Use $N=8$ in the \li{cheb(N)} method and use barycentric interpolation to approximate $u$ on 100 evenly spaced points.
Compare your numerical solution with the exact solution,
\[
u(x) = \frac{- \cosh(2) - \sinh(2)x + e^{2x}}{4}.
\]

	Use the pseudospectral method to solve the boundary value problem
	\begin{align*}
	&{ } u'' + u' = e^{3x}, \quad x \in (-1,1), \\
	&{ } u(-1) = 2, \quad u(1) = -1.
	\end{align*}
	
	Use $N=8$ in the \li{cheb(N)} method and use barycentric interpolation to approximate $u$ on 100 evenly spaced points.

Use the pseudospectral method to solve the boundary value problem
\begin{align*}
&{ } u'' = \lambda\sinh(\lambda u), \quad x \in (0,1), \\
&{ } u(0) = 0, \quad u(1) = 1
\end{align*}
for several values of $\lambda$: $\lambda = 4, 8, 12$. 
Begin by transforming this BVP onto the domain $-1<x<1$.
Use $N=20$ in the \li{cheb(N)} method and use barycentric interpolation to approximate $u$ on 100 evenly spaced points.
 
Below is sample code for implementing Newton's Method
\begin{lstlisting}
from scipy.optimize import root

N = 20
D, x = cheb(20)

def F(U):
	out = None	#Set up the equation you want the root of.
	#Make sure to set the boundaries correctly
	
	return out	#Newtons Method will update U until the output is all 0's.

guess = None    #Make your guess, same size as the cheb(N) output
solution = root(F, guess).x 
\end{lstlisting}

Find the function $y(x)$ that satisfies $y(-1) = 1$, $y(1) = 7$, and whose surface of revolution (about the $x$-axis) minimizes surface area.
Compute the surface area, and plot the surface. \label{prob:pseudospectral1_revision:minimal_surface}
Use $N=50$ in the \li{cheb(N)} method and use barycentric interpolation to approximate $u$ on 100 evenly spaced points.

Below is sample code for creating the 3D wireframe figure.
\begin{lstlisting}
from mpl_toolkits.mplot3d import Axes3D

barycentric = None	#This is the output of barycentric_interpolate() on 100 points

lin = np.linspace(-1, 1, 100)
theta = np.linspace(0,2*np.pi,401)
X, T = np.meshgrid(lin, theta)
Y, Z = barycentric*np.cos(T), barycentric*np.sin(T)

fig = plt.figure()
ax = fig.gca(projection="3d")
ax.plot_wireframe(X, Y, Z, rstride=10, cstride=10)
plt.show()
\end{lstlisting}

