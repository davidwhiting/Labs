
Solve and graph the Lorenz equation by completing the code.
Initialize the initial conditions with random values between $-15$ and $15$.
For this exercise, let $\sigma=10$, $\rho=28$, $\beta=\frac{8}{3}$.
Compare to Figure \ref{fig:Single_Lorenz}
\begin{lstlisting}
import numpy as np
from scipy.integrate import odeint

def lorenz_ode(inputs, T):
	'''
	Code up the sytem of equations given
	'''
	return Xprime, Yprime, Zprime

def solve_lorenx(init_cond, time=10):
	T = np.linspace(0, time, time*100)	#initialize time interval for ode
	'''
	Use odeint in conjuction with lorenz_ode and the time interval T
	To get the X, Y, and Z values for this system.
	You will need to transpose the output of odeint to graph it correctly.
	'''
 	return X, Y, Z

sigma = 'value'
rho = 'value'
beta = 'value'
init_cond = [x0, y0, z0]

X, Y, Z = solve_lorenz(init_cond, 50)
'''
Code to graph
'''
\end{lstlisting}


Change your code to plot $n$ different solution using different random initial conditions.
Produce a plot with $n=3$ different solutions.
Compare to Figure \ref{fig:Multiple_Lorenz}

Change the code above so that it initializes one set of initial conditions and creates a second set of initial conditions by adding the array \li{np.random.randn(3)*(1E-10)}.
This will represent a small perturbation in the initial conditions. Make sure the \li{time} variable is large enough to notice a difference in the two solutions.
Plot both solutions together.
Refer to Figure \ref{fig:perturbed_lorenz}

Animate the solutions of the Lorenz equation for an initial set of conditions and the perturbed conditions on the same plot.
To make the animation go faster, decrease the \li{interval} value in the \li{FuncAnimation()} call.
It will take several seconds before the curves split, so be patient.

Now set one initial condition.
Use \li{odeint} to solve the system, but use the arguments \li{atol=1E-14}, \li{rtol=1E-12}, and then again with \li{atol=1E-15}, \li{rtol=1E-13}.
Animate both solutions on the same plot.

Write a Python function that finds an initial point on the strange attractor, runs the simulation to a given time $t$, and produces a semilog plot of the norm of the difference between the two solution curves.
Also have it plot an exponential line fitted to match the curve (this will be linear on the semilog plot).
Have it return a rough estimate of the Lyapunov exponent.
The output should be something like Figure \ref{fig:lyapunov_exponent}.

Note: In order to get a good estimate of the Lyapunov exponent, your initial guess should already lie on the strange attractor.
You can get a value on the attractor by running the system for a while to find a good initial guess.

Hint: To find the fitting line, take the logarithm of the norms of the differences, compute a linear fit, then take the exponential function of the resulting line.
The Lyapunov exponent will be approximately equal to the slope found by the linear regression.
