 The solution of
\begin{align}
	\begin{split}
		y' &= y - 2x + 4,\quad 0 \leq x \leq 2, \\
		y(0) &= 0,
	\end{split}\label{ivp:prob1}
\end{align}
is given by $y(x) = -2+2x + 2e^x.$
Use Euler's method to numerically approximate the solution with step sizes $h = 0.2, 0.1$, and $0.05.$ 
Implement the following code to initialize variables and compute all $y$ values.

\begin{lstlisting}
def initialize_all(a,b,y0,h):
    """Given an initial and final time a and b, with y(a)=y0, and step size h,
    return several things.
    
    X: an aray from a to b with n elements, where n is the number of steps from a to b.
    Y: an empty array of size (n, y.size), Y[0]=y0.
    h: the step size.
    n: the number of steps to be taken.
    
    """
    n = int((b-a)/h+1)
    X = np.linspace(a, b, n)
    if isinstance(y0, np.ndarray):
        Y = np.empty((n, y0.size))
    else:
        Y = np.empty(n)
    Y[0] = y0
    return X, Y, h, int(n)

def euler(f,X,Y,h,n):
    """Use the Euler method to compute an approximate solution
    to the ODE y' = f(t, y) over X.

    Y[0] = y0
    f is assumed to accept two arguments.
    The first is a constant giving the value of t.
    The second is a one-dimensional numpy array of the same size as y.

    This function returns an array Y of shape (n,) if
    y is a constant or an array of size 1.
    It returns an array of shape (n, y.size) otherwise.
    In either case, Y[i] is the approximate value of y at
    the i'th value of X.
    """
    
    return None
\end{lstlisting}

Graph the results and check that your results match Figure \ref{ivp:euler}.

Suppose a differential equation is given by
\[ y' = f(t).\]
Since the function $f$ has no $y$ dependence, this a simple integration problem. 
Which integration approximation methods correspond to Euler's method, backward Euler's method, modified Euler's method, the Midpoint method, and the fourth order Runge-Kutta method (RK4)?
 Consider the IVP \eqref{ivp:prob2}.
Use the Midpoint method and the fourth order Runge-Kutta method (RK4) to approximate the value of the solution at $x = 2$, with a step size of $h = 0.2,$ $ 0.1,$ $0.05 $, $0.025,$ and $0.0125.$ 
Create a log-log plot of the relative error of each approximation using the \li{loglog} function in \li{matplotlib} (see Figure \ref{ivp:relative_error}).
 Use the RK4 method to solve for the simple harmonic oscillator satisfying 
\begin{align}
	\begin{split}
&{}my'' + ky = 0,\quad 0 \leq t \leq 20, \\
&{}y(0) = 2, \quad
y'(0) = -1,
	\end{split}
	\label{ivp:simple_oscillator}
\end{align}
for $m = 1$ and $k =1$. Note that in your implementation of RK4, the constants $K_1, K_2, K_3,$ and $K_4$ become vectors with $n$ entries, where $n$ is the number of equations in the first-order system.

Plot your numerical approximation of $y(t)$.  
Compare this with its numerical approximation when $m = 3$ and $k =1$. Consider: Why does the difference in solutions make sense physically?

Use the RK4 method to solve for the damped free harmonic oscillator satisfying 
\begin{align*}
&{}y'' +\gamma y'+ y = 0, \quad 0 \leq t \leq 20,\\
&{}y(0) = 1, \quad
y'(0) = -1.
\end{align*}
For $\gamma = 1/2,$ and $\gamma = 1$, simultaneously plot your numerical approximations of $y$.  
Print $y(20)$ accurate to four significant digits, by checking that the relative error is less than $5\times 10^{-5}$.

Use the RK4 method to solve for the damped and forced harmonic oscillator satisfying 
\begin{align}
	\begin{split}
&{}2y'' + \gamma y' + 2y = 2 \cos (\omega t), \quad 0 \leq t \leq 40,\\
&{}y(0) = 2, \quad
y'(0) = -1. 
	\end{split}
	\label{ivp:damped_forced_oscillator}
\end{align}
For the following values of $\gamma$ and $\omega,$ plot your numerical approximations of $y$ and print $y(40)$ accurate to four significant digits: $(\gamma, \omega) = (0.5, 1.5),$ $(0.1, 1.1),$ and $(0, 1).$
