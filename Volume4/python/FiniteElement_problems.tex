
Use the finite element method to solve
\begin{align}
	\begin{split}
	&{ }\epsilon y'' - y' = -1,\\
	&{ }y(0) = \alpha, \quad y(1) = \beta,
	\end{split} \label{eqn:FEM_exercise}
\end{align}
where $\alpha = 2, \beta = 4$, and $\epsilon = 0.02$.
Use $N = 100$ finite elements ($101$ grid points).
Compare your solution with the analytic solution
\[y(x) = \alpha + x + (\beta - \alpha - 1 ) \frac{e^{x/\epsilon} -1}{e^{1/\epsilon} -1}.\]

This boundary value problem has a singularly perturbed ODE, because the parameter $\epsilon$ is a coefficient of the highest order derivative in the equation.
The character of the problem changes dramatically when $\epsilon = 0$: since the limit equation $y' = 1$ is first-order, it only allows for one boundary condition.
Thus as $\epsilon$ gets smaller, the rightmost boundary condition is satisfied at the `last moment',  and cannot be satisfied when $\epsilon = 0$.

One of the strengths of the finite element method is the ability to generate grids that better suit the problem.
In two dimensions the finite elements are quadrilaterals and triangles, and can be used to approximate irregular domains.
The finite element method can also be used to solve PDEs where the shape and size of the domain changes over time.

The solution of \eqref{eqn:FEM_exercise} changes most rapidly near $x = 1$.
Compare the numerical solution when the grid points are unevenly spaced versus when the grid points are clustered in the area of greatest change; see Figure \ref{fig:FEM_compare_methods}. Specifically, use the grid points defined by
\begin{lstlisting}
even_grid = np.linspace(0,1,15)
clustered_grid = np.linspace(0,1,15)**(1./8)
\end{lstlisting}
What is the difference in accuracy?

\label{prob:FEM_accuracy_comparison}
Higher order methods promise faster convergence, but typically require more work to code.
So why do we use them when a low order method will converge just as well, albeit with more grid points?
The answer concerns the roundoff error associated with floating point arithmetic.
Low order methods generally require more floating point operations, so roundoff error has a much greater effect.

The finite element method introduced here is a second order method, even though the approximate solution is piecewise linear.
(To see this, note that if the grid points are evenly spaced, the matrix $A$ in \eqref{eqn:FEM_linear_system} is exactly the same as the matrix for the second order centered finite difference method.)

Solve \eqref{eqn:FEM_exercise} with the finite element method using $N = 2^i$  finite elements, $i = 4, 5, \ldots, 21$.
Use a log-log plot to graph the error; see Figure \ref{fig:FEM_error_2nd_order}.
