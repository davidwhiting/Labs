
The \li{pydataset} module\footnote{Run \texttt{pip install pydataset} if needed.} contains numerous data sets, each stored as a pandas \li{DataFrame}.

\begin{lstlisting}
>>> from pydataset import data

# Call data() to see the entire list of data sets.
# To load a particular data set, enter its ID as an argument to data().
>>> titanic = data("Titanic")
# To see the information about a data set, call data() with show_doc=True.
>>> data("Titanic", show_doc=True)
<<Titanic

PyDataset Documentation (adopted from R Documentation. The displayed
examples are in R)

## Survival of passengers on the Titanic>>
\end{lstlisting}

Visualize and describe at least 5 of the following data sets with 2 or 3 figures each.
Comment on the implications and significance of each visualization and give a comprehensive summary of the data set.

\begin{itemize}
\item \li{"Arbuthnot"}: Ratios of male to female births in London from 1629-1710
\item \li{"trees"}: Girth, height and volume for black cherry trees
\item \li{"road"}: Road accident deaths in the United States
\item \li{"birthdeathrates"}: Birth and death rates by country
\item \li{"bfeed"}: Child breast feeding records
\item \li{"heart"}: Survival of patients on the waiting list for the Stanford heart transplant program
\item \li{"lung"}: Survival in patients with advanced lung cancer from the North Central Cancer Treatment group
\item \li{"birthwt"}: Risk factors associated with low infant birth weight
\item A data set of your choice
\end{itemize}
Include each of the following in each visualization.
\begin{itemize}
\item A clear title, with relevant information for the period or region the data was collected in.
\item Axis labels that specify units.
\item A legend (if appropriate).
\item The source.
You may include the source information in your plot or print it after the plot.
\end{itemize}
