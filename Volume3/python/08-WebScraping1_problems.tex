 % Use View Page Source to see HTML source of a simple website.
The HTML of a website is easy to view in most browsers.
In Google Chrome, go to \url{http://www.example.com}, right click anywhere on the page that isn't a picture or a link, and select \li{View Page Source}.
This will open the HTML source code that defines the page.
Examine the source code.
What tags are used?
What is the value of the \li{<<type>>} attribute associated with the \li{style} tag?

Write a function that returns the set of names of tags used in the website, and the value of the \li{<<type>>} attribute of the \li{style} tag (as a string).
\\(Hint: there are ten unique tag names.)
\label{prob:look-at-example.com}
 % Find tag names using BeautifulSoup.
The \li{BeautifulSoup} class has a \li{find_all()} method that, when called with \li{True} as the only argument, returns a list of all tags in the HTML source code.

Write a function that accepts a string of HTML code as an argument.
Use BeautifulSoup to return a list of the \textbf{names} of the tags in the code.
Use your function and the source code from \url{http://www.example.com} (see \texttt{example.html}) to check your answers from Problem \ref{prob:look-at-example.com}.

The file \texttt{example.html} contains the HTML source for \url{http://www.example.com}.
Write a function that reads the file and loads the code into BeautifulSoup.
Find the only \li{<a>} tag with a hyperlink and return its text.
 % TODO: give them the html file, don't download it.

% example.html
Write a function that returns the following line using three different methods.
\begin{lstlisting}
<<'More information...'>>
\end{lstlisting}
The function should accept an integer.
If the integer is 1, find the line using tag names and the \li{.string} method.
If the integer is 2, find the line by traversing through the children of the body tag with repeated calls to \li{.contents}.
If the integer is 3, find the line by using navigation between siblings and \li{.string}.

The file \texttt{san\_diego\_weather.html} contains the HTML source for an old page from Weather Underground.\footnote{See \url{http://www.wunderground.com/history/airport/KSAN/2015/1/1/DailyHistory.html?req_city=San+Diego&req_state=CA&req_statename=California&reqdb.zip=92101&reqdb.magic=1&reqdb.wmo=99999&MR=1}}.
Write a function that reads the file and loads it into BeautifulSoup.
Return a list of the following tags:
\begin{enumerate}
\item The tag containing the date ``Thursday, January 1, 2015''.
\item The tags which contain the \textbf{links} ``Previous Day'' and ``Next Day.''
\item The tag which contains the number associated with the Actual Max Temperature.
\end{enumerate}

This HTML tree is significantly larger than the previous examples.
To get started, consider opening the file in a web browser.
Find the element that you are searching for on the page, right click it, and select \li{Inspect}.
This opens the HTML source at the element that the mouse clicked on.
 % Bank Data Index.
The file \texttt{large\_banks\_index.html} is an index of data about large banks, as recorded by the Federal Reserve.\footnote{See \url{https://www.federalreserve.gov/releases/lbr/}.}
Write a function that reads the file and loads the source into BeautifulSoup.
Return a list of the tags containing the links to bank data from September 30, 2003 to December 31, 2014, where the dates are in reverse chronological order.
\label{prob:bs-bank-index}
 % Actual Bank Data.
The file \texttt{large\_banks\_data.html} is one of the pages from the index in Problem \ref{prob:bs-bank-index}.\footnote{See \url{http://www.federalreserve.gov/releases/lbr/20030930/default.htm}.}
Write a function that reads the file and loads the source into BeautifulSoup.
Create a single figure with two subplots:
\begin{enumerate}
    \item A sorted bar chart of the seven banks with the most domestic branches.
    \item A sorted bar chart of the seven banks with the most foreign branches.
\end{enumerate}
In the case of a tie, sort the banks alphabetically by name.
