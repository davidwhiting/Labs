 % Create a student database.
Write a function that accepts the name of a database file.
Connect to the database (and create it if it doesn't exist).
Drop the tables \li{MajorInfo}, \li{CourseInfo}, \li{StudentInfo}, and \li{StudentGrades} from the database \textbf{if} they exist.
Next, add the following tables to the database with the specified column names and types.
\begin{itemize}
\item \li{MajorInfo}: \li{MajorID} (integers) and \li{MajorName} (strings).
\item \li{CourseInfo}: \li{CourseID} (integers) and \li{CourseName} (strings).
\item \li{StudentInfo}: \li{StudentID} (integers), \li{StudentName} (strings), and \li{MajorID} (integers).
\item \li{StudentGrades}: \li{StudentID} (integers), \li{CourseID} (integers), and \li{Grade} (strings).
\end{itemize}
Remember to commit and close the database.
You should be able to execute your function more than once with the same input without raising an error.

To check the database, use the following commands to get the column names of a specified table.
Assume here that the database file is called \texttt{students.db}.

\begin{lstlisting}
>>> with sql.connect("students.db") as conn:
...     cur = conn.cursor()
...     cur.execute("SELECT * FROM StudentInfo;")
...     print([d[0] for d in cur.description])
...
<<['StudentID', 'StudentName', 'MajorID']>>
\end{lstlisting}
\label{prob:sql1-create-student-tables}
 % Populate student tables.
Expand your function from Problem \ref{prob:sql1-create-student-tables} so that it populates the tables with the data given in Tables \ref{table:sql1-student-majorinfo}--\ref{table:sql1-student-grades}.

\begin{table}[H]
\begin{subtable}{0.55\textwidth}
    \centering
    \begin{subtable}{.45\textwidth}
        \centering
        \footnotesize
        \begin{tabular}{|l|l|}
            \hline MajorID & MajorName \\ \hline
            1 & Math \\
            2 & Science \\
            3 & Writing \\
            4 & Art \\ \hline
        \end{tabular}
        \caption{MajorInfo}
        \label{table:sql1-student-majorinfo}
    \end{subtable}
    \hfil
    \begin{subtable}{.45\textwidth}
        \centering
        \footnotesize
        \begin{tabular}{|l|l|}
            \hline CourseID & CourseName \\ \hline
            1 & Calculus \\
            2 & English \\
            3 & Pottery \\
            4 & History \\ \hline
        \end{tabular}
        \caption{CourseInfo}
        \label{table:sql1-student-courseinfo}
    \end{subtable}
    \\[1.em] % NOTE: Use 4.01em to align the tables at the top and bottom.
    \begin{subtable}{\textwidth}
        \centering
        \footnotesize
        \begin{tabular}{|l|l|l|}
            \hline StudentID & StudentName & MajorID \\ \hline
            401767594 & Michelle Fernandez & 1 \\
            678665086 & Gilbert Chapman & NULL \\
            553725811 & Roberta Cook & 2 \\
            886308195 & Rene Cross & 3 \\
            103066521 & Cameron Kim & 4 \\
            821568627 & Mercedes Hall & NULL \\
            206208438 & Kristopher Tran & 2 \\
            341324754 & Cassandra Holland & 1 \\
            262019426 & Alfonso Phelps & NULL \\
            622665098 & Sammy Burke & 2 \\ \hline
        \end{tabular}
        \caption{StudentInfo}
        \label{table:sql1-student-info}
    \end{subtable}
\end{subtable}
\hfil
\begin{subtable}{.35\textwidth}
    \centering
    \footnotesize
    \begin{tabular}{|l|l|l|}
        \hline StudentID & CourseID & Grade \\ \hline
        401767594 & 4 & C \\
        401767594 & 3 & B-- \\
        678665086 & 4 & A+ \\
        678665086 & 3 & A+ \\
        553725811 & 2 & C \\
        678665086 & 1 & B \\
        886308195 & 1 & A \\
        103066521 & 2 & C \\
        103066521 & 3 & C-- \\
        821568627 & 4 & D \\
        821568627 & 2 & A+ \\
        821568627 & 1 & B \\
        206208438 & 2 & A \\
        206208438 & 1 & C+ \\
        341324754 & 2 & D-- \\
        341324754 & 1 & A-- \\
        103066521 & 4 & A \\
        262019426 & 2 & B \\
        262019426 & 3 & C \\
        622665098 & 1 & A \\
        622665098 & 2 & A-- \\ \hline
    \end{tabular}
    \caption{StudentGrades}
    \label{table:sql1-student-grades}
\end{subtable}
\caption{Student database.}
\end{table}

The \li{StudentInfo} and \li{StudentGrades} tables are also recorded in \texttt{student\_info.csv} and \texttt{student\_grades.csv}, respectively, with \lsql{NULL} values represented as $-1$.
A CSV (\textbf{c}omma-\textbf{s}eparated \textbf{v}alues) file can be read like a normal text file or with the \li{csv} module.

\begin{lstlisting}
>>> import csv
>>> with open("student_info.csv", 'r') as infile:
...     rows = list(csv.reader(infile))
\end{lstlisting}

To validate your database, use the following command to retrieve the rows from a table.

\begin{lstlisting}
>>> with sql.connect("students.db") as conn:
...     cur = conn.cursor()
...     for row in cur.execute("SELECT * FROM MajorInfo;"):
...         print(row)
<<(1, 'Math')
(2, 'Science')
(3, 'Writing')
(4, 'Art')>>
\end{lstlisting}
\label{prob:sql1-populate-students}

The data file \texttt{us\_earthquakes.csv}\footnote{Retrieved from \url{https://datarepository.wolframcloud.com/resources/Sample-Data-US-Earthquakes}.} contains data from about 3,500 earthquakes in the United States since the 1769.
Each row records the year, month, day, hour, minute, second, latitude, longitude, and magnitude of a single earthquake (in that order).
Note that latitude, longitude, and magnitude are floats, while the remaining columns are integers.

Write a function that accepts the name of a database file.
Drop the table \li{USEarthquakes} if it already exists, then create a new \li{USEarthquakes} table with schema \li{(Year, Month, Day, Hour, Minute, Second, Latitude, Longitude, Magnitude)}.
Populate the table with the data from \texttt{us\_earthquakes.csv}.
Remember to commit the changes and close the connection.
\\ (Hint: using \li{executemany()} is much faster than using \li{execute()} in a loop.)
\label{prob:sql1-create-earthquakes}
 % Dealing with missing data.
Modify your function from Problems \ref{prob:sql1-create-student-tables} and \ref{prob:sql1-populate-students} so that in the \li{StudentInfo} table, values of $-1$ in the \li{MajorID} column are replaced with \lsql{NULL} values.

Also modify your function from Problem \ref{prob:sql1-create-earthquakes} in the following ways.
\begin{enumerate}
    \item Remove rows from \li{USEarthquakes} that have a value of $0$ for the \li{Magnitude}.
    \item Replace $0$ values in the \li{Day}, \li{Hour}, \li{Minute}, and \li{Second} columns with \lsql{NULL} values.
\end{enumerate}
 % SELECT data from multiple tables.
Write a function that accepts the name of a database file.
Assuming the database to be in the format of the one created in Problems \ref{prob:sql1-create-student-tables} and \ref{prob:sql1-populate-students}, query the database for all tuples of the form (\li{StudentName}, \li{CourseName}) where that student has an ``A'' or ``A+'' grade in that course.
Return the list of tuples.

Write a function that accepts the name of a database file.
Assuming the database to be in the format of the one created in Problem \ref{prob:sql1-create-earthquakes}, query the \li{USEarthquakes} table for the following information.
\begin{itemize}
    \item The magnitudes of the earthquakes during the 19th century ($1800$--$1899$).
    \item The magnitudes of the earthquakes during the 20th century ($1900$--$1999$).
    \item The average magnitude of all earthquakes in the database.
\end{itemize}
Create a single figure with two subplots: a histogram of the magnitudes of the earthquakes in the 19th century, and a histogram of the magnitudes of the earthquakes in the 20th century.
Show the figure, then return the average magnitude of all of the earthquakes in the database.
Be sure to return an actual number, not a list or a tuple.
\\(Hint: use \li{np.ravel()} to convert a result set of 1-tuples to a 1-D array.)
\label{prob:sql1-earthquake-analysis}
