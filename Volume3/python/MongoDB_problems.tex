 % Load a collection with trump-related tweets.
The file \texttt{trump.json} contains posts from \url{http://www.twitter.com} (tweets) over the course of an hour that have the key word ``trump''.\footnote{See the Additional Materials section for an example of using the Twitter API.}
Each line in the file is a single JSON message that can be loaded with \li{json.loads()}.

Create a MongoDB database and initialize a collection in the database.
Use the collection's \li{delete_many()} method with an empty set as input to clear existing contents of the collection, then fill the collection one line at a time with the data from \texttt{trump.json}.
Check that your collection has 95,643 entries with its \li{count()} method.
\label{prob:mongo-fill-db}

Query the Twitter collection from Problem \ref{prob:mongo-fill-db} for the following information.
\begin{itemize}
    \item How many tweets include the word Russia? Use \li{re.IGNORECASE}.
    \item How many tweets came from one of the main continental US time zones? These are listed as \li{"Central Time (US \& Canada)"}, \li{"Pacific Time (US \& Canada)"}, \li{"Eastern Time (US \& Canada)"}, and \li{"Mountain Time (US \& Canada)"}.
    \item How often did each language occur? Construct a dictionary with each language and it's frequency count.
    \\(Hint: use \li{distinct()} to get the language options.)
\end{itemize}

Query the Twitter collection from Problem \ref{prob:mongo-fill-db} for the following information.
\begin{itemize}
    \item What are the usernames of the 5 most popular (defined as having the most followers) tweeters? Don't include repeats.
    \item Of the tweets containing at least 5 hashtags, sort the tweets by how early the 5th hashtag appears in the text.
    What is the earliest spot (character count) it appears?
    \item What are the coordinates of the tweet that came from the northernmost location?
    Use the latitude and longitude point in \li{"coordinates"}.
\end{itemize}

Clean the twitter collection in the following ways.
\begin{itemize}
\item Get rid of the \li{"retweeted_status"} field in each tweet.

\item Update every tweet from someone with at least 1000 followers to include a \li{popular} field whose value is \li{True}.
Report the number of popular tweets.

\item (OPTIONAL) The geographical coordinates used before in \li{coordinates.coordinates} are turned off for most tweets. But many more have a bounding box around the coordinates in the \li{place} field. Update every tweet without coordinates that contains a bounding box so that the coordinates contains the average value of the points that form the bounding box. Make the structure of \li{coordinates} the same as the others, so it contains \li{coordinates} with a longitude, latitude array and a \li{type}, the value of which should be 'Point'.
\\(Hint: Iterate through each tweet in with a bounding box but no coordinates. Then for each tweet, grab it's id and the bounding box coordinates. Find the average, and then update the tweet. To update it search for it's id and then give the needed update parameters. First unset coordinates, and then set coordinates.coordinates and coordinates.type to the needed values.)
\end{itemize}

The file \li{mylans_bistro.json} contains a json object describing one additional restaurant.  Insert it into the collection. Note that this entry contains an additional key value not present in any other.  A SQL database would have to be entirely rebuilt to support this insertion, but with MongoDB this is not an issue.

After this insert, use a query to list every restaurant that closes at eighteen o'clock (Mylan's Bistro should be one of these).

Query your new collection to answer the following questions:
\begin{itemize}
\item How many of the restaurants are in Manhattan?
\item How many restaurants have gotten a grade other than an ``A" on a health inspection?
\item Which are the ten northernmost restaurants?
\item Which restaurants have ``grill" (case-insensitive) in their names?
\end{itemize}

Use update operators to perform the following tasks:
\begin{itemize}
\item Whenever a restaurant has ``grill" in its name, replace ``grill" with ``Magical Fire Table".
\item Increase all of the restaurant IDs by 1000.
\item Delete the entries of every restaurant that has ever gotten a ``C" health inspection grade.
\end{itemize}
