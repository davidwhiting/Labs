
Create a pandas \li{Series} where the index labels are the even integers $0,2,\ldots,50$, and the entries are $n^2 - 1$, where $n$ is the entry's label.
Set all of the entries equal to zero whose labels are divisible by $3$.

Suppose you make an investment of $d$ dollars in a particularly volatile stock.
Every day the value of your stock goes up by \$$1$ with probability $p$, or down by \$$1$ with probability $1-p$ (this is an example of a \emph{random walk}).

Write a function that accepts a probability parameter $p$ and an initial amount of money $d$, defaulting to $100$.
Use \li{pd.date_range()} to create an index of the days from 1 January 2000 to 31 December 2000.
Simulate the daily change of the stock by making one draw from a Bernoulli distribution with parameter $p$ (a binomial distribution with one draw) for each day.
Store the draws in a pandas \li{Series} with the date index and set the first draw to the initial amount $d$.
Sum the entries cumulatively to get the stock value by day.
Set any negative values to $0$, then plot the series using the \li{plot()} method of the \li{Series} object.

Call your function with a few different values of $p$ and $d$ to observe the different possible kinds of behavior.
\label{prob:pandas-random-walk}

Create five random walks of length 100, and plot them together in the same plot.

Next, create a ``biased'' random walk by changing the coin flip probability of head from 0.5 to 0.51.
Plot this biased walk with lengths 100, 10000, and then 100000.
Notice the definite trend that emerges.
Your results should be comparable to those in Figure \ref{pandas:biasedRandomWalk}.

The example above shows how to implement a simple WHERE condition, and it is easy
to have a more complex expression.
Simply enclose each condition by parentheses,
and use the standard boolean operators \li{\&} (AND), \li{\|} (OR), and \li{\~} (NOT) to
connect the conditions appropriately.
Use pandas to execute the following query:
\begin{lstlisting}[language=SQL]
SELECT ID, Name from studentInfo WHERE Age > 19 AND Sex = 'M'
\end{lstlisting}

Using a join operation, create a \li{DataFrame} containing the ID, age, and GPA of all male individuals.
You ought to be able to accomplish this in one line of code.
 % Crime data.
The file \texttt{crime\_data.csv} contains data on types of crimes committed in the United States from 1960 to 2016.
\begin{itemize}
\item Load the data into a pandas \li{DataFrame}, using the column names in the file and the column titled
``Year'' as the index.
Make sure to skip lines that don't contain data.

\item Insert a new column into the data frame that contains the crime rate by year (the ratio of ``Total'' column
to the ``Population'' column).

\item Plot the crime rate as a function of the year.

\item List the 5 years with the highest crime rate in descending order.

\item Calculate the average number of total crimes as well as burglary crimes between 1960 and 2012.

\item Find the years for which the total number of crimes was below average, but the number of burglaries
was above average.

\item Plot the number of murders as a function of the population.

\item Select the Population, Violent, and Robbery columns for all years in the 1980s, and save
this smaller data frame to a CSV file \texttt{crime\_subset.csv}.
\end{itemize}

In 1912 the RMS \emph{Titanic} sank after colliding with an iceberg.
The file \texttt{titanic.csv} contains data on the incident.
Each row represents a different passenger, and the columns describe various features of the passengers (age, sex, whether or not they survived, etc.)

Start by cleaning the data.
\begin{itemize}
    \item Read the data into a \li{DataFrame}.
    Use the first row of the file as the column labels, but do not use any of the columns as the index.
    \item Drop the columns \li{"Sibsp"}, \li{"Parch"}, \li{"Cabin"}, \li{"Boat"}, \li{"Body"}, and \li{"home.dest"}.
    \item Drop any entries without data in the \li{"Survived"} column, then change the remaining entries to \li{True} or \li{False} (they start as 1 or 0).
    \item Replace null entries in the \li{"Age"} column with the average age.
    \item Save the new \li{DataFrame} as \texttt{titanic\_clean.csv}.
\end{itemize}
Next, answer the following questions.
\begin{itemize}
    \item How many people survived? What percentage of passengers survived?
    \item What was the average price of a ticket? How much did the most expensive ticket cost?
    \item How old was the oldest survivor? How young was the youngest survivor? What about non-survivors?
\end{itemize}



