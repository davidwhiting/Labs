 % Check feasibility at the origin.
Write a class that accepts the arrays $\c$, $A$, and $\b$ of a linear optimization problem in standard form.
In the constructor, check that the system is feasible at the origin\footnote{For now, we only check for feasibility at the origin.
A more robust solver sets up the auxiliary problem and solves it to find a starting point if the origin is infeasible.}.
That is, check that $A\x \preceq \b$ when $\x = \0$.
Raise a \li{ValueError} if the problem is not feasible at the origin.
\label{prob:initsolver}
 % Slack variables. % TODO: this problem is worthless...
Design and implement a way to store and track all of the basic and non-basic variables.

Hint: Using integers that represent the index of each variable is useful for Problem \ref{prob:blands}.
\label{prob:slackvars}
 % Initialize the tableau.
Add a method to your Simplex solver that will create the initial tableau as a NumPy array.
\label{prob:maketableau}

Write a method that will determine the pivot row and pivot column according to Bland's Rule.
% \begin{comment}
\begin{definition}[Bland's Rule]
Choose the nonbasic variable with the smallest index that has a positive coefficient in the objective function
as the leaving variable.
Choose the basic variable with the smallest index among all the binding basic variables.
\end{definition}

Bland's Rule is important in avoiding cycles when performing pivots.
This rule guarantees that a feasible Simplex problem will terminate in a finite number of pivots.
% \end{comment}
\label{prob:blands}
 % Pivoting
Add a method to your solver that checks for unboundedness and performs a single pivot operation from start to completion.
If the problem is unbounded, raise a \li{ValueError}.

Write an additional method in your solver called \li{solve()} that obtains the optimal solution, then returns the maximum value, the basic variables, and the nonbasic variables.
The basic and nonbasic variables should be represented as two dictionaries that map the index of the variable to its corresponding value.

For our example, we would return the tuple \li{(5.2, \{0: 1.6, 1: .2, 2: .6\}, \{3: 0, 4: 0\})}.
%The correct format of this tuple is critical, as this tuple of information will be used judge whether or not your solver works!
 % Product mix problem.
Solve the product mix problem for the data contained in the file \li{productMix.npz}.
In this problem, there are 4 products and 3 resources.
The archive file, which you can load using the function
\li{np.load}, contains a dictionary of arrays.
The array with key \li{'A'} gives the resource coefficients $a_{i,j}$ (i.e. the $(i,j)$-th entry of the array give $a_{i,j}$).
The array with key \li{'p'} gives the unit prices $p_i$.
The array with key \li{'m'} gives the available resource
units $m_j$.
The array with key \li{'d'} gives the demand constraints $d_i$.

Report the number of units that should be produced for each product.

% What is the final tableau for this Klee-Minty example with $n=3$?
% How many iterations does it take to arrive at the final tableau?
% 
% Using problem 1 as a guide, guess the optimum value of the Klee-Minty example with $n=20$.
% Then find the maximum value for the Klee-Minty example with $n=20$.
% How many iterations does it take to arrive at the final tableau?
% How long does the program take to run for only $20$ variables?
% 