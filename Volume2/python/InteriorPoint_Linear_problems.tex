 % First step: define F(x,lamb,mu) -----------------------------
Define a function \li{interiorPoint()} that will be used to solve the complete interior point problem.
This function should accept $A$, $\b$, and $\c$ as parameters, along with the keyword arguments \li{niter=20} and \li{tol=1e-16}.
The keyword arguments will be used in a later problem.

For this problem, within the \li{interiorPoint()} function, write a function for the vector-valued function $F$ described above.
This function should accept $\x$, $\Lamb$, and $\Mu$ as parameters and return a 1-dimensional NumPy array with $2n+m$ entries.
 % Search direction --------------------------------------------
Within \li{interiorPoint()}, write a subroutine to compute the search direction $(\triangle \x^T, \triangle \Lamb^T, \triangle \Mu^T)$ by solving Equation \ref{eq:newNewton}.
Use $\sigma = \frac{1}{10}$ for the centering parameter.

Note that only the last block row of $DF$ will need to be changed at each iteration (since $M$ and $X$ depend on $\Mu$ and $\x$, respectively).
Consider using the functions \li{lu_factor()} and \li{lu_solve()} from the \li{scipy.linalg} module to solving the system of equations efficiently.
 % Step size ---------------------------------------------------
Within \li{interiorPoint()}, write a subroutine to compute the step size after the search direction has been computed.
Avoid using loops when computing $\alpha_{\max}$ and $\beta_{\max}$ (use masking and NumPy functions instead).
 % Put it all together -----------------------------------------
Complete the implementation of \li{interiorPoint()}.

Use the function \li{starting_point()} provided above to select an initial point, then run the iteration \li{niter} times, or until the duality measure is less than \li{tol}.
Return the optimal point $\x^*$ and the optimal value $\c^T\x^*$.

The duality measure $\nu$ tells us in some sense how close our current point is to the minimizer.
The closer $\nu$ is to 0, the closer we are to the optimal point.
Thus, by printing the value of $\nu$ at each iteration, you can track how your algorithm is progressing and detect when you have converged.

To test your implementation, use the following code to generate a random linear program, along with the optimal solution.
% TODO: update the spec file with this version of randomLP().
\begin{lstlisting}
    """Generate a linear program min c^T x s.t. Ax = b, x>=0.
    First generate m feasible constraints, then add
    slack variables to convert it into the above form.
    Inputs:
        m (int >= n): number of desired constraints.
        n (int): dimension of space in which to optimize.
    Outputs:
        A ((m,n+m) ndarray): Constraint matrix.
        b ((m,) ndarray): Constraint vector.
        c ((n+m,), ndarray): Objective function with m trailing 0s.
        x ((n,) ndarray): The first 'n' terms of the solution to the LP.
    """
    A = np.random.random((m,n))*20 - 10
    A[A[:,-1]<0] *= -1
    x = np.random.random(n)*10
    b = np.zeros(m)
    b[:n] = A[:n,:].dot(x)
    b[n:] = A[n:,:].dot(x) + np.random.random(m-n)*10
    c = np.zeros(n+m)
    c[:n] = A[:n,:].sum(axis=0)/n
    A = np.hstack((A, np.eye(m)))
    return A, b, -c, x
\end{lstlisting}
\begin{lstlisting}
>>> m, n = 7, 5
>>> A, b, c, x = randomLP(m, n)
>>> point, value = interiorPoint(A, b, c)
>>> np.allclose(x, point[:n])
True
\end{lstlisting}
 % Least Total Deviations Problem ------------------------------
The file \li{simdata.txt} contains two columns of data.
The first gives the values of the response variables ($y_i$), and the second column gives the values of the explanatory variables ($\x_i$).
Find the least absolute deviations line for this data set, and plot it together with the data.
Plot the least squares solution as well to compare the results.
\begin{lstlisting}
>>> from scipy.stats import linregress
>>> slope, intercept = linregress(data[:,1], data[:,0])[:2]
>>> domain = np.linspace(0,10,200)
>>> plt.plot(domain, domain*slope + intercept)
\end{lstlisting}
% \begin{figure}[H] % solution to the problem with new simdata.
% \centering
% \includegraphics[width=\textwidth]{figures/LADprob.pdf}
% \label{fig:LADprob}
% \end{figure}
