 % Restricting data types of the Node class.
Consider the following generic node class.
\begin{lstlisting}
class Node:
    """A basic node class for storing data."""
    def __init__(self, data):
        """Store 'data' in the 'value' attribute."""
        self.value = data
\end{lstlisting}

Modify the constructor so that it only accepts data of type \li{int}, \li{float}, or \li{str}.
If another type of data is given, raise a \li{TypeError} with an appropriate error message.
Modify the constructor docstring to document these restrictions.

Add a method called \li{find(self, data)} to the \li{LinkedList} class that
returns the first node in the list containing \li{data} (return the actual \li{LinkedListNode} object, not its \li{value}).
If no such node exists, or if the list is empty, raise a \li{ValueError} with an appropriate error message.
\\
(Hint: if \li{current} is assigned to one of the nodes the list, what does the following line do?)
\begin{lstlisting}
current = current.<<next>>
\end{lstlisting}
 % __len__() and __str__() for the LinkedList class.
Add magic methods to the \li{LinkedList} class so it behaves more like the built-in Python list.
\begin{enumerate}
\item Write the \li{__len__()} method so that the length of a \li{LinkedList} instance is equal to the number of nodes in the list.
To accomplish this, consider adding an attribute that tracks the current size of the list.
It should be updated every time a node is successfully added or removed.

\item Write the \li{__str__()} method so that when a \li{LinkedList} instance is printed, its output matches that of a Python list.
Entries are separated by a comma and one space, and strings are surrounded by single quotes.
Note the difference between the string representations of the following lists:

\begin{lstlisting}
>>> num_list = [1, 2, 3]
>>> str_list = ['1', '2', '3']
>>> print(num_list)
[1, 2, 3]
>>> print(str_list)
<<['1', '2', '3']>>
\end{lstlisting}
\end{enumerate}
 % LinkedList.remove().
Modify the \li{remove()} method given above so that it correctly removes the first node in the list containing the specified data.
Account for the special cases of removing the first, last, or only node.
 % LinkedList.insert()
Add a method called \li{insert(self, data, place)} to the \li{LinkedList} class that inserts a new node containing \li{data} immediately before the first node in the list containing \li{place}.
Account for the special case of inserting before the first node.

See Figure \ref{fig:insert} for an illustration.
Note that since \li{insert()} places a new node before an existing node, it is not possible to use \li{insert()} to put a new node at the end of the list or in an empty list (use \li{append()} instead).
 % Deque class.
Write a \li{Deque} class that inherits from the \li{LinkedList} class.
%
\begin{enumerate}
\item Use inheritance to implement the following methods:
%
\begin{itemize}
    \item \li{pop(self)}: Remove the last node in the list and return its data.
    \item \li{popleft(self)}: Remove the first node in the list and return its data.
    \item \li{appendleft(self, data)}: Insert a new node containing \li{data} at the beginning of the list.
\end{itemize}
The \li{LinkedList} class already implements the \li{append()} method.

\item Override the \li{remove()} method with the following:

\begin{lstlisting}
def remove(*args, **kwargs):
    raise NotImplementedError("Use pop() or popleft() for removal")
\end{lstlisting}

This effectively disables \li{remove()} for the \li{Deque} class, preventing the user from removing a node from the middle of the list.

\item Disable the \li{insert()} method as well.
\end{enumerate}
 % Reverse a file using a stack/deque.
Write a function that accepts the name of a file to be read and a file to write to.
Read the first file, adding each line to the end of a deque.
After reading the entire file, pop each entry off of the end of the deque one at a time, writing the result to a line of the second file.

For example, if the file to be read has the list of words on the left, the resulting file should have the list of words on the right.

\begin{lstlisting}
<<My homework is too hard for me.         I am a mathematician.
I do not believe that                   Programming is hard, but
I can solve these problems.             I can solve these problems.
Programming is hard, but                I do not believe that
I am a mathematician.                   My homework is too hard for me.
\end{lstlisting}

You may use a Python list, your \li{Deque} class, or \li{collections.deque} for the deque.
Test your function on the file \texttt{english.txt}, which contains a list of over 58,000 English words in alphabetical order.
