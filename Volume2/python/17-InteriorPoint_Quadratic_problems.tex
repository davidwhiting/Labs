
Copy your \li{interiorPoint()} function from the previous lab into your solutions file for this lab, renaming it \li{qInteriorPoint()}.
This new function should accept the arrays $Q, \c, A,$ and $\b$, a tuple of arrays \li{guess} giving initial estimates for $\x, \y,$ and $\Mu$ (this will be explained later), along with the keyword arguments \li{niter=20} and \li{tol=1e-16}.

Modify your code to match the $F$ and $DF$ described above, and calculate the search direction $(\triangle \x^T, \triangle \y^T, \triangle \Mu^T)$ by solving Equation \ref{eq:searchDirection}.
Use $\sigma = \frac{1}{10}$ for the centering parameter.

Hint: What are the dimensions of $F$ and $DF$?

Complete the implementation of \li{qInteriorPoint()}.
Return the optimal point $\x$ as well as the final objective function value.
You may want to print out the duality measure $\nu$ to check the progress of the iteration.
% and \li{verbose}, a boolean value indicating whether or not to print the current objective function value ($ \frac{1}{2}x_k^TQx_k + c^Tx_k$) and duality measure ($\nu_k = \frac{1}{m}y_k^T\lambda_k$) at each iteration.
% 
Solve the circus tent problem with the tent pole configuration given above, for grid side length $n = 15$.
Plot your solution.

The text file \li{portfolio.txt} contains historical stock data for several assets (U.S. bonds, gold, S\&P 500, etc).
In particular, the first column gives the years corresponding to the data, and the remaining eight columns give the historical returns
of eight assets over the course of these years.
Use this data to estimate the covariance matrix $Q$ as well as the expected rates of return $\mu_i$ for each asset.
Assuming that we want to guarantee an expected return of $R = 1.13$ for our portfolio, find the optimal portfolio both with and without short selling.

Since the problem contains both equality and inequality constraints, use the QP solver in CVXOPT rather than your \li{qInteriorPoint()} function.

Hint: Use \li{numpy.cov()} to compute Q.
